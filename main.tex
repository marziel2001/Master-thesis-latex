\documentclass{article}
\usepackage[T1]{fontenc} % for polish characters

\title{Praca magisterska - Porównanie i analiza modeli Speech To Text}
\author{Marcel Zieliński}
\date{ }

\begin{document}

\maketitle
\break

\tableofcontents
\break


\addcontentsline{toc}{section}{Unnumbered Section}
\section*{Unnumbered Section}
sekcja

\section{Second Section}
% Spis treści
\tableofcontents
\newpage

Cytat przykładowy \cite{zielonka2023survey}.

% Rozdziały z osobnych plików
% Wstęp
\section{Wstęp}
To jest przykładowy wstęp.

% Metodyka
\section{Metodyka}
Opis metodyki badań.

% Wyniki
\section{Wyniki}
Prezentacja wyników.
Hehehehe
% Podsumowanie
\section{Podsumowanie}
Podsumowanie pracy.


% Bibliography
\bibliographystyle{plain}
\bibliography{bibliography}

\end{document}

% SPIS TREŚCI
% Wstęp
%     Uzasadnienie wyboru tematu
%     Cel i zakres pracy
%     Metodyka badań i narzędzia
%     Struktura pracy

% Rozdział 1. Teoretyczne podstawy technologii Speech-to-Text
% 1.1. Wprowadzenie do technologii rozpoznawania mowy
% 1.2. Ewolucja systemów ASR (Automatic Speech Recognition)
% 1.3. Kluczowe pojęcia: fonemy, modele akustyczne, modele językowe
% 1.4. Metody uczenia modeli STT (HMM, DNN, Transformer, Whisper)
% 1.5. Standardy i protokoły wykorzystywane w systemach STT
% 1.6. Wyzwania technologiczne: szumy, akcenty, kontekst, modele on-device
% Rozdział 2. Przegląd dostępnych narzędzi i usług Speech-to-Text
% 2.1. Kryteria porównawcze narzędzi STT
% 2.2. Google Speech-to-Text API
% 2.3. Microsoft Azure Speech Service
% 2.4. Amazon Transcribe
% 2.5. OpenAI Whisper (on-device i w chmurze)
% 2.6. Web Speech API (przeglądarkowe)
% 2.7. Porównanie dostępnych narzędzi:
%    2.7.1. Dokładność transkrypcji
%    2.7.2. Opóźnienia i czas przetwarzania
%    2.7.3. Wymagania sprzętowe
%    2.7.4. Koszty użytkowania
%    2.7.5. Możliwość pracy offline
%    2.7.6. Dostępność języka polskiego
% Rozdział 3. Technologia React jako platforma implementacji
% 3.1. Architektura React – komponenty, hooki, state management
% 3.2. Integracja z API i usługami zewnętrznymi
% 3.3. Komunikacja w czasie rzeczywistym (WebSocket, Streaming, REST)
% 3.4. Obsługa audio w przeglądarce (MediaDevices, AudioContext, MediaRecorder)
% 3.5. Wymagania wydajnościowe po stronie frontendu
% Rozdział 4. Projekt i implementacja aplikacji React wykorzystującej Speech-to-Text
% 4.1. Założenia funkcjonalne aplikacji
% 4.2. Projekt architektury systemu
%    4.2.1. Ogólny schemat przepływu danych
%    4.2.2. Integracja z wybranymi usługami STT
%    4.2.3. Warstwa frontendu (React)
%    4.2.4. Ewentualna warstwa backendowa (Node.js / Express)
% 4.3. Implementacja modułu nagrywania dźwięku
% 4.4. Implementacja komunikacji z API STT
% 4.5. Wizualizacja i edycja transkrypcji
% 4.6. Obsługa błędów i fallbacki (np. Whisper offline)
% 4.7. Testy funkcjonalne i wydajnościowe
% 4.8. Wyzwania implementacyjne i sposoby ich rozwiązania
% Rozdział 5. Studium porównawcze skuteczności wybranych narzędzi STT
% 5.1. Projekt eksperymentu
%    5.1.1. Przygotowanie zestawu nagrań
%    5.1.2. Warunki badania i pomiarów
% 5.2. Metryki oceny jakości transkrypcji (WER, CER, RTT)
% 5.3. Wyniki testów porównawczych
% 5.4. Analiza statystyczna wyników
% 5.5. Wnioski z badań
% Rozdział 6. Ocena przydatności technologii STT w aplikacjach webowych
% 6.1. Zastosowania praktyczne
% 6.2. Skalowalność i efektywność kosztowa
% 6.3. Bezpieczeństwo i prywatność danych audio
% 6.4. Rekomendacje implementacyjne
% 6.5. Ograniczenia i możliwe kierunki rozwoju
% Zakończenie

%     Podsumowanie pracy
%     Najważniejsze osiągnięcia
%     Kierunki dalszych badań